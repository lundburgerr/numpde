
% Default to the notebook output style

    


% Inherit from the specified cell style.




    
\documentclass[11pt]{article}

    
    
    \usepackage[T1]{fontenc}
    % Nicer default font than Computer Modern for most use cases
    \usepackage{palatino}

    % Basic figure setup, for now with no caption control since it's done
    % automatically by Pandoc (which extracts ![](path) syntax from Markdown).
    \usepackage{graphicx}
    % We will generate all images so they have a width \maxwidth. This means
    % that they will get their normal width if they fit onto the page, but
    % are scaled down if they would overflow the margins.
    \makeatletter
    \def\maxwidth{\ifdim\Gin@nat@width>\linewidth\linewidth
    \else\Gin@nat@width\fi}
    \makeatother
    \let\Oldincludegraphics\includegraphics
    % Set max figure width to be 80% of text width, for now hardcoded.
    \renewcommand{\includegraphics}[1]{\Oldincludegraphics[width=.8\maxwidth]{#1}}
    % Ensure that by default, figures have no caption (until we provide a
    % proper Figure object with a Caption API and a way to capture that
    % in the conversion process - todo).
    \usepackage{caption}
    \DeclareCaptionLabelFormat{nolabel}{}
    \captionsetup{labelformat=nolabel}

    \usepackage{adjustbox} % Used to constrain images to a maximum size 
    \usepackage{xcolor} % Allow colors to be defined
    \usepackage{enumerate} % Needed for markdown enumerations to work
    \usepackage{geometry} % Used to adjust the document margins
    \usepackage{amsmath} % Equations
    \usepackage{amssymb} % Equations
    \usepackage{textcomp} % defines textquotesingle
    % Hack from http://tex.stackexchange.com/a/47451/13684:
    \AtBeginDocument{%
        \def\PYZsq{\textquotesingle}% Upright quotes in Pygmentized code
    }
    \usepackage{upquote} % Upright quotes for verbatim code
    \usepackage{eurosym} % defines \euro
    \usepackage[mathletters]{ucs} % Extended unicode (utf-8) support
    \usepackage[utf8x]{inputenc} % Allow utf-8 characters in the tex document
    \usepackage{fancyvrb} % verbatim replacement that allows latex
    \usepackage{grffile} % extends the file name processing of package graphics 
                         % to support a larger range 
    % The hyperref package gives us a pdf with properly built
    % internal navigation ('pdf bookmarks' for the table of contents,
    % internal cross-reference links, web links for URLs, etc.)
    \usepackage{hyperref}
    \usepackage{longtable} % longtable support required by pandoc >1.10
    \usepackage{booktabs}  % table support for pandoc > 1.12.2
    \usepackage[normalem]{ulem} % ulem is needed to support strikethroughs (\sout)
                                % normalem makes italics be italics, not underlines
    
    \DeclareMathOperator{\Div}{div}
\DeclareMathOperator{\Grad}{grad}
\DeclareMathOperator{\Curl}{curl}
\DeclareMathOperator{\Rot}{rot}
\DeclareMathOperator{\ord}{ord}
\DeclareMathOperator{\Kern}{ker}
\DeclareMathOperator{\Image}{im}
\DeclareMathOperator{\spann}{span}
\DeclareMathOperator{\dist}{dist}
\DeclareMathOperator{\diam}{diam}
\DeclareMathOperator{\sig}{sig}
\newcommand{\RR}{\mathbb{R}}
\newcommand{\NN}{\mathbb{N}}
\newcommand{\VV}{\mathbb{V}}
\newcommand{\dGamma}{\,\mathrm{d} \Gamma}
\newcommand{\dGammah}{\,\mathrm {d} \Gamma_h}
\newcommand{\dx}{\,\mathrm{d}x}
\newcommand{\dy}{\,\mathrm{d}y}
\newcommand{\ds}{\,\mathrm{d}s}
\newcommand{\dt}{\,\mathrm{d}t}
\newcommand{\dS}{\,\mathrm{d}S}
\newcommand{\dV}{\,\mathrm{d}V}
\newcommand{\dX}{\,\mathrm{d}X}
\newcommand{\dY}{\,\mathrm{d}Y}
\newcommand{\dE}{\,\mathrm{d}E}
\newcommand{\dK}{\,\mathrm{d}K}
\newcommand{\dM}{\,\mathrm{d}M}
\newcommand{\cd}{\mathrm{cd}}
\newcommand{\onehalf}{\frac{1}{2}}
\newcommand{\bfP}{\boldsymbol P}
\newcommand{\bfx}{\boldsymbol x}
\newcommand{\bfa}{\boldsymbol a}
\newcommand{\bfu}{\boldsymbol u}
\newcommand{\bfv}{\boldsymbol v}
\newcommand{\bfe}{\boldsymbol e}
\newcommand{\bfb}{\boldsymbol b}
\newcommand{\bff}{\boldsymbol f}
\newcommand{\bfp}{\boldsymbol p}
\newcommand{\bft}{\boldsymbol t}
\newcommand{\bfj}{\boldsymbol j}
\newcommand{\bfB}{\boldsymbol B}
\newcommand{\bfV}{\boldsymbol V}
\newcommand{\bfE}{\boldsymbol E}

    
    % Colors for the hyperref package
    \definecolor{urlcolor}{rgb}{0,.145,.698}
    \definecolor{linkcolor}{rgb}{.71,0.21,0.01}
    \definecolor{citecolor}{rgb}{.12,.54,.11}

    % ANSI colors
    \definecolor{ansi-black}{HTML}{3E424D}
    \definecolor{ansi-black-intense}{HTML}{282C36}
    \definecolor{ansi-red}{HTML}{E75C58}
    \definecolor{ansi-red-intense}{HTML}{B22B31}
    \definecolor{ansi-green}{HTML}{00A250}
    \definecolor{ansi-green-intense}{HTML}{007427}
    \definecolor{ansi-yellow}{HTML}{DDB62B}
    \definecolor{ansi-yellow-intense}{HTML}{B27D12}
    \definecolor{ansi-blue}{HTML}{208FFB}
    \definecolor{ansi-blue-intense}{HTML}{0065CA}
    \definecolor{ansi-magenta}{HTML}{D160C4}
    \definecolor{ansi-magenta-intense}{HTML}{A03196}
    \definecolor{ansi-cyan}{HTML}{60C6C8}
    \definecolor{ansi-cyan-intense}{HTML}{258F8F}
    \definecolor{ansi-white}{HTML}{C5C1B4}
    \definecolor{ansi-white-intense}{HTML}{A1A6B2}

    % commands and environments needed by pandoc snippets
    % extracted from the output of `pandoc -s`
    \providecommand{\tightlist}{%
      \setlength{\itemsep}{0pt}\setlength{\parskip}{0pt}}
    \DefineVerbatimEnvironment{Highlighting}{Verbatim}{commandchars=\\\{\}}
    % Add ',fontsize=\small' for more characters per line
    \newenvironment{Shaded}{}{}
    \newcommand{\KeywordTok}[1]{\textcolor[rgb]{0.00,0.44,0.13}{\textbf{{#1}}}}
    \newcommand{\DataTypeTok}[1]{\textcolor[rgb]{0.56,0.13,0.00}{{#1}}}
    \newcommand{\DecValTok}[1]{\textcolor[rgb]{0.25,0.63,0.44}{{#1}}}
    \newcommand{\BaseNTok}[1]{\textcolor[rgb]{0.25,0.63,0.44}{{#1}}}
    \newcommand{\FloatTok}[1]{\textcolor[rgb]{0.25,0.63,0.44}{{#1}}}
    \newcommand{\CharTok}[1]{\textcolor[rgb]{0.25,0.44,0.63}{{#1}}}
    \newcommand{\StringTok}[1]{\textcolor[rgb]{0.25,0.44,0.63}{{#1}}}
    \newcommand{\CommentTok}[1]{\textcolor[rgb]{0.38,0.63,0.69}{\textit{{#1}}}}
    \newcommand{\OtherTok}[1]{\textcolor[rgb]{0.00,0.44,0.13}{{#1}}}
    \newcommand{\AlertTok}[1]{\textcolor[rgb]{1.00,0.00,0.00}{\textbf{{#1}}}}
    \newcommand{\FunctionTok}[1]{\textcolor[rgb]{0.02,0.16,0.49}{{#1}}}
    \newcommand{\RegionMarkerTok}[1]{{#1}}
    \newcommand{\ErrorTok}[1]{\textcolor[rgb]{1.00,0.00,0.00}{\textbf{{#1}}}}
    \newcommand{\NormalTok}[1]{{#1}}
    
    % Additional commands for more recent versions of Pandoc
    \newcommand{\ConstantTok}[1]{\textcolor[rgb]{0.53,0.00,0.00}{{#1}}}
    \newcommand{\SpecialCharTok}[1]{\textcolor[rgb]{0.25,0.44,0.63}{{#1}}}
    \newcommand{\VerbatimStringTok}[1]{\textcolor[rgb]{0.25,0.44,0.63}{{#1}}}
    \newcommand{\SpecialStringTok}[1]{\textcolor[rgb]{0.73,0.40,0.53}{{#1}}}
    \newcommand{\ImportTok}[1]{{#1}}
    \newcommand{\DocumentationTok}[1]{\textcolor[rgb]{0.73,0.13,0.13}{\textit{{#1}}}}
    \newcommand{\AnnotationTok}[1]{\textcolor[rgb]{0.38,0.63,0.69}{\textbf{\textit{{#1}}}}}
    \newcommand{\CommentVarTok}[1]{\textcolor[rgb]{0.38,0.63,0.69}{\textbf{\textit{{#1}}}}}
    \newcommand{\VariableTok}[1]{\textcolor[rgb]{0.10,0.09,0.49}{{#1}}}
    \newcommand{\ControlFlowTok}[1]{\textcolor[rgb]{0.00,0.44,0.13}{\textbf{{#1}}}}
    \newcommand{\OperatorTok}[1]{\textcolor[rgb]{0.40,0.40,0.40}{{#1}}}
    \newcommand{\BuiltInTok}[1]{{#1}}
    \newcommand{\ExtensionTok}[1]{{#1}}
    \newcommand{\PreprocessorTok}[1]{\textcolor[rgb]{0.74,0.48,0.00}{{#1}}}
    \newcommand{\AttributeTok}[1]{\textcolor[rgb]{0.49,0.56,0.16}{{#1}}}
    \newcommand{\InformationTok}[1]{\textcolor[rgb]{0.38,0.63,0.69}{\textbf{\textit{{#1}}}}}
    \newcommand{\WarningTok}[1]{\textcolor[rgb]{0.38,0.63,0.69}{\textbf{\textit{{#1}}}}}
    
    
    % Define a nice break command that doesn't care if a line doesn't already
    % exist.
    \def\br{\hspace*{\fill} \\* }
    % Math Jax compatability definitions
    \def\gt{>}
    \def\lt{<}
    % Document parameters
    \title{lecture\_01}
    
    
    

    % Pygments definitions
    
\makeatletter
\def\PY@reset{\let\PY@it=\relax \let\PY@bf=\relax%
    \let\PY@ul=\relax \let\PY@tc=\relax%
    \let\PY@bc=\relax \let\PY@ff=\relax}
\def\PY@tok#1{\csname PY@tok@#1\endcsname}
\def\PY@toks#1+{\ifx\relax#1\empty\else%
    \PY@tok{#1}\expandafter\PY@toks\fi}
\def\PY@do#1{\PY@bc{\PY@tc{\PY@ul{%
    \PY@it{\PY@bf{\PY@ff{#1}}}}}}}
\def\PY#1#2{\PY@reset\PY@toks#1+\relax+\PY@do{#2}}

\expandafter\def\csname PY@tok@gp\endcsname{\let\PY@bf=\textbf\def\PY@tc##1{\textcolor[rgb]{0.00,0.00,0.50}{##1}}}
\expandafter\def\csname PY@tok@ow\endcsname{\let\PY@bf=\textbf\def\PY@tc##1{\textcolor[rgb]{0.67,0.13,1.00}{##1}}}
\expandafter\def\csname PY@tok@gr\endcsname{\def\PY@tc##1{\textcolor[rgb]{1.00,0.00,0.00}{##1}}}
\expandafter\def\csname PY@tok@ge\endcsname{\let\PY@it=\textit}
\expandafter\def\csname PY@tok@cp\endcsname{\def\PY@tc##1{\textcolor[rgb]{0.74,0.48,0.00}{##1}}}
\expandafter\def\csname PY@tok@vg\endcsname{\def\PY@tc##1{\textcolor[rgb]{0.10,0.09,0.49}{##1}}}
\expandafter\def\csname PY@tok@ss\endcsname{\def\PY@tc##1{\textcolor[rgb]{0.10,0.09,0.49}{##1}}}
\expandafter\def\csname PY@tok@vi\endcsname{\def\PY@tc##1{\textcolor[rgb]{0.10,0.09,0.49}{##1}}}
\expandafter\def\csname PY@tok@sb\endcsname{\def\PY@tc##1{\textcolor[rgb]{0.73,0.13,0.13}{##1}}}
\expandafter\def\csname PY@tok@sx\endcsname{\def\PY@tc##1{\textcolor[rgb]{0.00,0.50,0.00}{##1}}}
\expandafter\def\csname PY@tok@kp\endcsname{\def\PY@tc##1{\textcolor[rgb]{0.00,0.50,0.00}{##1}}}
\expandafter\def\csname PY@tok@s1\endcsname{\def\PY@tc##1{\textcolor[rgb]{0.73,0.13,0.13}{##1}}}
\expandafter\def\csname PY@tok@w\endcsname{\def\PY@tc##1{\textcolor[rgb]{0.73,0.73,0.73}{##1}}}
\expandafter\def\csname PY@tok@sd\endcsname{\let\PY@it=\textit\def\PY@tc##1{\textcolor[rgb]{0.73,0.13,0.13}{##1}}}
\expandafter\def\csname PY@tok@c1\endcsname{\let\PY@it=\textit\def\PY@tc##1{\textcolor[rgb]{0.25,0.50,0.50}{##1}}}
\expandafter\def\csname PY@tok@na\endcsname{\def\PY@tc##1{\textcolor[rgb]{0.49,0.56,0.16}{##1}}}
\expandafter\def\csname PY@tok@nc\endcsname{\let\PY@bf=\textbf\def\PY@tc##1{\textcolor[rgb]{0.00,0.00,1.00}{##1}}}
\expandafter\def\csname PY@tok@m\endcsname{\def\PY@tc##1{\textcolor[rgb]{0.40,0.40,0.40}{##1}}}
\expandafter\def\csname PY@tok@gh\endcsname{\let\PY@bf=\textbf\def\PY@tc##1{\textcolor[rgb]{0.00,0.00,0.50}{##1}}}
\expandafter\def\csname PY@tok@kc\endcsname{\let\PY@bf=\textbf\def\PY@tc##1{\textcolor[rgb]{0.00,0.50,0.00}{##1}}}
\expandafter\def\csname PY@tok@cs\endcsname{\let\PY@it=\textit\def\PY@tc##1{\textcolor[rgb]{0.25,0.50,0.50}{##1}}}
\expandafter\def\csname PY@tok@ni\endcsname{\let\PY@bf=\textbf\def\PY@tc##1{\textcolor[rgb]{0.60,0.60,0.60}{##1}}}
\expandafter\def\csname PY@tok@mh\endcsname{\def\PY@tc##1{\textcolor[rgb]{0.40,0.40,0.40}{##1}}}
\expandafter\def\csname PY@tok@k\endcsname{\let\PY@bf=\textbf\def\PY@tc##1{\textcolor[rgb]{0.00,0.50,0.00}{##1}}}
\expandafter\def\csname PY@tok@gs\endcsname{\let\PY@bf=\textbf}
\expandafter\def\csname PY@tok@mo\endcsname{\def\PY@tc##1{\textcolor[rgb]{0.40,0.40,0.40}{##1}}}
\expandafter\def\csname PY@tok@c\endcsname{\let\PY@it=\textit\def\PY@tc##1{\textcolor[rgb]{0.25,0.50,0.50}{##1}}}
\expandafter\def\csname PY@tok@s2\endcsname{\def\PY@tc##1{\textcolor[rgb]{0.73,0.13,0.13}{##1}}}
\expandafter\def\csname PY@tok@nt\endcsname{\let\PY@bf=\textbf\def\PY@tc##1{\textcolor[rgb]{0.00,0.50,0.00}{##1}}}
\expandafter\def\csname PY@tok@o\endcsname{\def\PY@tc##1{\textcolor[rgb]{0.40,0.40,0.40}{##1}}}
\expandafter\def\csname PY@tok@gd\endcsname{\def\PY@tc##1{\textcolor[rgb]{0.63,0.00,0.00}{##1}}}
\expandafter\def\csname PY@tok@gi\endcsname{\def\PY@tc##1{\textcolor[rgb]{0.00,0.63,0.00}{##1}}}
\expandafter\def\csname PY@tok@nn\endcsname{\let\PY@bf=\textbf\def\PY@tc##1{\textcolor[rgb]{0.00,0.00,1.00}{##1}}}
\expandafter\def\csname PY@tok@mf\endcsname{\def\PY@tc##1{\textcolor[rgb]{0.40,0.40,0.40}{##1}}}
\expandafter\def\csname PY@tok@se\endcsname{\let\PY@bf=\textbf\def\PY@tc##1{\textcolor[rgb]{0.73,0.40,0.13}{##1}}}
\expandafter\def\csname PY@tok@nd\endcsname{\def\PY@tc##1{\textcolor[rgb]{0.67,0.13,1.00}{##1}}}
\expandafter\def\csname PY@tok@nl\endcsname{\def\PY@tc##1{\textcolor[rgb]{0.63,0.63,0.00}{##1}}}
\expandafter\def\csname PY@tok@nv\endcsname{\def\PY@tc##1{\textcolor[rgb]{0.10,0.09,0.49}{##1}}}
\expandafter\def\csname PY@tok@sc\endcsname{\def\PY@tc##1{\textcolor[rgb]{0.73,0.13,0.13}{##1}}}
\expandafter\def\csname PY@tok@sr\endcsname{\def\PY@tc##1{\textcolor[rgb]{0.73,0.40,0.53}{##1}}}
\expandafter\def\csname PY@tok@go\endcsname{\def\PY@tc##1{\textcolor[rgb]{0.53,0.53,0.53}{##1}}}
\expandafter\def\csname PY@tok@kn\endcsname{\let\PY@bf=\textbf\def\PY@tc##1{\textcolor[rgb]{0.00,0.50,0.00}{##1}}}
\expandafter\def\csname PY@tok@cm\endcsname{\let\PY@it=\textit\def\PY@tc##1{\textcolor[rgb]{0.25,0.50,0.50}{##1}}}
\expandafter\def\csname PY@tok@kt\endcsname{\def\PY@tc##1{\textcolor[rgb]{0.69,0.00,0.25}{##1}}}
\expandafter\def\csname PY@tok@s\endcsname{\def\PY@tc##1{\textcolor[rgb]{0.73,0.13,0.13}{##1}}}
\expandafter\def\csname PY@tok@si\endcsname{\let\PY@bf=\textbf\def\PY@tc##1{\textcolor[rgb]{0.73,0.40,0.53}{##1}}}
\expandafter\def\csname PY@tok@err\endcsname{\def\PY@bc##1{\setlength{\fboxsep}{0pt}\fcolorbox[rgb]{1.00,0.00,0.00}{1,1,1}{\strut ##1}}}
\expandafter\def\csname PY@tok@bp\endcsname{\def\PY@tc##1{\textcolor[rgb]{0.00,0.50,0.00}{##1}}}
\expandafter\def\csname PY@tok@mb\endcsname{\def\PY@tc##1{\textcolor[rgb]{0.40,0.40,0.40}{##1}}}
\expandafter\def\csname PY@tok@sh\endcsname{\def\PY@tc##1{\textcolor[rgb]{0.73,0.13,0.13}{##1}}}
\expandafter\def\csname PY@tok@gu\endcsname{\let\PY@bf=\textbf\def\PY@tc##1{\textcolor[rgb]{0.50,0.00,0.50}{##1}}}
\expandafter\def\csname PY@tok@cpf\endcsname{\let\PY@it=\textit\def\PY@tc##1{\textcolor[rgb]{0.25,0.50,0.50}{##1}}}
\expandafter\def\csname PY@tok@ne\endcsname{\let\PY@bf=\textbf\def\PY@tc##1{\textcolor[rgb]{0.82,0.25,0.23}{##1}}}
\expandafter\def\csname PY@tok@vc\endcsname{\def\PY@tc##1{\textcolor[rgb]{0.10,0.09,0.49}{##1}}}
\expandafter\def\csname PY@tok@nb\endcsname{\def\PY@tc##1{\textcolor[rgb]{0.00,0.50,0.00}{##1}}}
\expandafter\def\csname PY@tok@no\endcsname{\def\PY@tc##1{\textcolor[rgb]{0.53,0.00,0.00}{##1}}}
\expandafter\def\csname PY@tok@nf\endcsname{\def\PY@tc##1{\textcolor[rgb]{0.00,0.00,1.00}{##1}}}
\expandafter\def\csname PY@tok@kd\endcsname{\let\PY@bf=\textbf\def\PY@tc##1{\textcolor[rgb]{0.00,0.50,0.00}{##1}}}
\expandafter\def\csname PY@tok@il\endcsname{\def\PY@tc##1{\textcolor[rgb]{0.40,0.40,0.40}{##1}}}
\expandafter\def\csname PY@tok@kr\endcsname{\let\PY@bf=\textbf\def\PY@tc##1{\textcolor[rgb]{0.00,0.50,0.00}{##1}}}
\expandafter\def\csname PY@tok@gt\endcsname{\def\PY@tc##1{\textcolor[rgb]{0.00,0.27,0.87}{##1}}}
\expandafter\def\csname PY@tok@ch\endcsname{\let\PY@it=\textit\def\PY@tc##1{\textcolor[rgb]{0.25,0.50,0.50}{##1}}}
\expandafter\def\csname PY@tok@mi\endcsname{\def\PY@tc##1{\textcolor[rgb]{0.40,0.40,0.40}{##1}}}

\def\PYZbs{\char`\\}
\def\PYZus{\char`\_}
\def\PYZob{\char`\{}
\def\PYZcb{\char`\}}
\def\PYZca{\char`\^}
\def\PYZam{\char`\&}
\def\PYZlt{\char`\<}
\def\PYZgt{\char`\>}
\def\PYZsh{\char`\#}
\def\PYZpc{\char`\%}
\def\PYZdl{\char`\$}
\def\PYZhy{\char`\-}
\def\PYZsq{\char`\'}
\def\PYZdq{\char`\"}
\def\PYZti{\char`\~}
% for compatibility with earlier versions
\def\PYZat{@}
\def\PYZlb{[}
\def\PYZrb{]}
\makeatother


    % Exact colors from NB
    \definecolor{incolor}{rgb}{0.0, 0.0, 0.5}
    \definecolor{outcolor}{rgb}{0.545, 0.0, 0.0}



    
    % Prevent overflowing lines due to hard-to-break entities
    \sloppy 
    % Setup hyperref package
    \hypersetup{
      breaklinks=true,  % so long urls are correctly broken across lines
      colorlinks=true,
      urlcolor=urlcolor,
      linkcolor=linkcolor,
      citecolor=citecolor,
      }
    % Slightly bigger margins than the latex defaults
    
    \geometry{verbose,tmargin=1in,bmargin=1in,lmargin=1in,rmargin=1in}
    
    

    \begin{document}
    
    
    \maketitle
    
    

    
    \section{Lecture 1. Course
Introduction}\label{lecture-1.-course-introduction}

(First, the boring but important adminstrative stuff!)

\subsubsection{Teachers}\label{teachers}

 Teacher

 Teacher Assistant

André Massing andre.massing@umu.se

Juan Carlos Araujo-Cabarcas juan.araujo@umu.se

    \subsubsection{Important dates}\label{important-dates}

\begin{itemize}
\tightlist
\item
  \textbf{January 2, 2017}: Last day for registration to the exam
\item
  \textbf{January 12, 2017}: Examination
\item
  \textbf{January 15, 2017} Deadline for submission of last lab report
\item
  \textbf{January 29, 2016} Deadline for resubmission of last lab report
\item
  \textbf{Februrary 15, 2017}: Last day for registration to the re-exam
\item
  \textbf{February 25, 2017}: Reexamination
\item
  Registration to exam always at least \textbf{10} days prior to exam
  day
\end{itemize}

    \subsubsection{Overall organization}\label{overall-organization}

\begin{itemize}
\tightlist
\item
  18 Lectures in total
\item
  Two last lectures in week 2: Revision of important concepts, Q\&A
  answers
\item
  5 Labs consisting of theoretical and practical problems

  \begin{itemize}
  \tightlist
  \item
    Practical problems can be solved in either Matlab,
    \href{https://www.gnu.org/software/octave/}{Octave} or
    \href{https://www.python.org/}{Python}
  \end{itemize}
\item
  3 half-day lab reservations a week, 2 of them will be supervised by
  Juan Carlos.
\item
  Live demos and example code in the lectures will use python packages
  from \href{https://www.scipy.org/}{SciPy ecosystem} and presented
  using Jupyter/IPython notebooks.
\item
  A crash course guiding you into the scientific Python world will be
  given in one of the supervised labs.
\end{itemize}

    \begin{Verbatim}[commandchars=\\\{\}]
{\color{incolor}In [{\color{incolor}2}]:} \PY{o}{\PYZpc{}\PYZpc{}}\PY{k}{HTML}
        \PYZlt{}iframe src=\PYZdq{}https://se.timeedit.net/web/umu/db1/public1/riq53Q03621Z6YQy7Q98Z6q667Z085087Y50Y0gQ00ogZ5QXQ6o.html\PYZdq{} height=600 width=1000\PYZgt{}
        \PYZlt{}/iframe\PYZgt{}
\end{Verbatim}

    
    \begin{verbatim}
<IPython.core.display.HTML object>
    \end{verbatim}

    
    \subsection{Reading material}\label{reading-material}

\begin{itemize}
\tightlist
\item
  Lectures, lecture notes, and notebooks for the part ``Introduction to
  Finite Difference Methods''
\item
  Book ``The Finite Element Method. Theory, Implementation and
  Applications'' by Mats G. Larson and Frederik Bengzon
\item
  Additionally (not required): ``Partial Differential Equations with
  Numerical Methods'' by Stig Larsson and Vidar Thomée
\item
  Collection of course material on
  \href{http://hplgit.github.io/num-methods-for-PDEs/doc/pub/}{``Numerical
  Methods for Partial Differential Equations''} by Hans-Petter
  Langtangen
\end{itemize}

    Beginning of next slide we add some \LaTeX macros inside a math
environment

\#\# What are Partial Differential Equations?

\textbf{Definition:} A partial differential equation is an equation
involving certaint partial derivatives of an unknown function
\(u: \Omega \subset \RR^d \to \RR\) with \(d \geqslant 2\).

Recall that \textbf{first order} partial derivatives

\begin{align*}
  D_i u(x) &=
 \partial_i u(x) =
 \partial_{x_i} u(x) =
 \dfrac{\partial}{\partial x_i} u(x) =
  u_{x_i}(x_0)=
             \lim_{h\to 0} \dfrac{u(x_0 + h e_i) - u(x_0)}{h}
             \\
             &=
  \lim_{h\to 0} \dfrac{
  u(x_1,\ldots,x_{i-1},x_i + h,x_{i+1},\ldots x_d) -
  u(x_1,\ldots,x_{i-1},x_i,x_{i+1},\ldots x_d)}{h}
\end{align*}
Second and higher order partial derivatives

\[
\partial_i \partial_j u = \dfrac{\partial^2}{\partial x_i \partial x_j}u
= D_i D_j u
\]

\[ D^{\alpha} u = 
\dfrac{\partial^{|\alpha|}}{\partial_1^{\alpha_1}\cdots\partial_d^{\alpha_d}} u
\]

for a multiindex \(\alpha = (\alpha_1, \ldots, \alpha_d)\) and
\(|\alpha| := \alpha_1 + \ldots + \alpha_d\)

    \subsection{Standard PDE operators}\label{standard-pde-operators}

\begin{itemize}
\tightlist
\item
  {\bf Gradient} of scalar function \(u: \Omega \subset \RR^d \to \RR\)
\end{itemize}

\[
\Grad u = \nabla u = \left( \partial_1 u, \ldots, \partial_d u \right)
\]

\begin{itemize}
\tightlist
\item
  {\bf Divergence} of a vector field
  \(\bfu: \Omega \subset \RR^d \to \RR^d\)
\end{itemize}

\[
\Div \bfu = \nabla \cdot \bfu = \sum_{i=1}^d \partial_i \bfu_i
\]

\begin{itemize}
\tightlist
\item
  {\bf Curl } of a vector field \(\bfu\)
\end{itemize}

\[
\Curl \bfu = \nabla \times \bfu =
\left(
\partial_y \bfu_z - \partial_z \bfu_y,
\partial_z \bfu_x - \partial_x \bfu_z,
\partial_x \bfu_y - \partial_y \bfu_x
\right)
\]

\begin{itemize}
\tightlist
\item
  {\bf Laplace} of \(u\)
\end{itemize}

\[
\Delta u =  \nabla \cdot (\nabla u) = \sum_{i=1}^d \partial_i u
\]

    \subsection{Examples for PDEs: Linear
equations}\label{examples-for-pdes-linear-equations}

\paragraph{Transport equation}\label{transport-equation}

Simplest ``real'' PDE for a function \(u(x,t)\) of two variables
\(x\in \RR\), \$ t \textgreater{} 0\$

\[
\partial_t u + b \partial_x u = f
\]

or more generally for \(u(x,t)\) with \(x \in \RR^d\)

\[
\partial_t u + \bfb \cdot \nabla u = f
\]

    \paragraph{Laplace's and Poisson's
equation}\label{laplaces-and-poissons-equation}

\[
 - \Delta u =  f
\] which often models equilibrium states in physics,
e.g.~electrostatistics, stationary heat distribution

    \paragraph{Heat equation}\label{heat-equation}

\[
\partial_tu - \Delta u = f
\]

    \paragraph{Wave equation}\label{wave-equation}

\[
\partial_t^2 u - \Delta u = f
\]

    \begin{Verbatim}[commandchars=\\\{\}]
{\color{incolor}In [{\color{incolor}3}]:} \PY{k+kn}{from} \PY{n+nn}{IPython}\PY{n+nn}{.}\PY{n+nn}{display} \PY{k}{import} \PY{n}{YouTubeVideo}\PY{p}{,} \PY{n}{HTML}
        \PY{n}{YouTubeVideo}\PY{p}{(}\PY{l+s+s1}{\PYZsq{}}\PY{l+s+s1}{nw2bPnhtxN8}\PY{l+s+s1}{\PYZsq{}}\PY{p}{,} \PY{n}{width}\PY{o}{=}\PY{l+m+mi}{1000}\PY{p}{,} \PY{n}{height}\PY{o}{=}\PY{l+m+mi}{500}\PY{p}{)}
        \PY{c+c1}{\PYZsh{} Alternative and more general way to embed general webpages}
        \PY{c+c1}{\PYZsh{}HTML(\PYZsq{}\PYZlt{}iframe width=\PYZdq{}1000\PYZdq{} height=\PYZdq{}500\PYZdq{} src=\PYZdq{}//www.youtube.com/embed/oWFSsCI3BzY\PYZdq{} frameborder=\PYZdq{}0\PYZdq{} allowfullscreen\PYZgt{}\PYZlt{}/iframe\PYZgt{}\PYZsq{})}
\end{Verbatim}
\texttt{\color{outcolor}Out[{\color{outcolor}3}]:}
    
    \begin{center}
    \adjustimage{max size={0.9\linewidth}{0.9\paperheight}}{lecture_01_files/lecture_01_12_0.jpeg}
    \end{center}
    { \hspace*{\fill} \\}
    

    \subsection{Nonlinear equations}\label{nonlinear-equations}

\paragraph{Korteweg-de Fries equation}\label{korteweg-de-fries-equation}

The propagation of waves in shallow waters is modeled by a 3rd order,
non-linear (actually, quasi-linear) PDE of the form \[
u_t - 6 u u_x + u_{xxx} = 0 
\]

    \paragraph{Minimal surface equation}\label{minimal-surface-equation}

A minimal surface \(\Gamma\subset \RR^3\) is a surface which locally
minimizes its surface area. Expressing the surface locally as the graph
of a function \(u(x,y)\), \(\Gamma\) being a minimal surface is
equivalent to satisfying the non-linear (elliptic) equation \[
\nabla \cdot \left(  
\dfrac{\nabla u}{ (1 + |\nabla u|^2)^{\onehalf}}
\right) = 0
\]

    \subsection{Linear systems of PDEs}\label{linear-systems-of-pdes}

\paragraph{Linear Elasticity}\label{linear-elasticity}

The displacement \(\bfu\) of a deformable medium at equilibrium subject
to an extern load \(\bfu\) can be modeled by \[
\mu\Delta\bfu  + (\lambda + \mu)\nabla (\nabla \cdot \bfu) = f
\]

The instationary case (basicially resembling Newton's second law
\(F = ma\)) is described through \[
\bfu_{tt} - \mu\Delta\bfu  - (\lambda + \mu)\nabla (\nabla \cdot \bfu) = f
\]

    \paragraph{Maxwell's equations}\label{maxwells-equations}

In the presence of a charge density \(\rho\) and a current density
\(\bfj\), the electric field \(\bfE\) and magnetic field \(\bfB\)
satisfy
\begin{align}
\Div \bfB &= 0 \\
\bfB_t + \Curl \bfE &= 0 \\
\Div \bfE &= 4\pi\rho \\
\bfE_t - \Curl \bfE &= - 4 \pi \bfj
\end{align}

    \subsection{Nonlinear systems of PDEs}\label{nonlinear-systems-of-pdes}

\paragraph{Reaction-Diffusion systems}\label{reaction-diffusion-systems}

\[
 \partial_t \bfu - \Delta \bfu = f(\bfu)
\]

for exampel the Gray-Scott equations describing
\begin{align}
\frac{\partial u}{\partial t} & = D_u \Delta u - u v^2 + F(1-u) \qquad \text{in } \Omega, 
\\  
\frac{\partial v}{\partial t} & = D_v \Delta v + u v^2 - (F+k)v \qquad \text{in } \Omega,     
\end{align} 

modeling the reaction and diffusion of chemical species \(U\) and \(V\)
described by their concentration \(u\) and \(v\). \(D_u\) and \(D_v\)
are the diffusion coefficients, \(k\) is the rate constant of the second
reaction and \(F\) the feed rate.

    

    \begin{Verbatim}[commandchars=\\\{\}]
{\color{incolor}In [{\color{incolor}4}]:} \PY{k+kn}{from} \PY{n+nn}{IPython}\PY{n+nn}{.}\PY{n+nn}{display} \PY{k}{import} \PY{n}{YouTubeVideo}\PY{p}{,} \PY{n}{HTML}
        \PY{n}{YouTubeVideo}\PY{p}{(}\PY{l+s+s1}{\PYZsq{}}\PY{l+s+s1}{nw2bPnhtxN8}\PY{l+s+s1}{\PYZsq{}}\PY{p}{,} \PY{n}{width}\PY{o}{=}\PY{l+m+mi}{1000}\PY{p}{,} \PY{n}{height}\PY{o}{=}\PY{l+m+mi}{500}\PY{p}{)}
\end{Verbatim}
\texttt{\color{outcolor}Out[{\color{outcolor}4}]:}
    
    \begin{center}
    \adjustimage{max size={0.9\linewidth}{0.9\paperheight}}{lecture_01_files/lecture_01_19_0.jpeg}
    \end{center}
    { \hspace*{\fill} \\}
    

    \paragraph{Navier-Stokes equations}\label{navier-stokes-equations}

describing incompressible, viscous flow in terms of the fluid velocity
\(\bfu\) and the fluid pressure \(p\)

\[
\partial_t \bfu + (\bfu \cdot \nabla) \bfu - \nu \Delta \bfu + \nabla p = \bff \\
\nabla \cdot \bfu = 0
\]

    \begin{Verbatim}[commandchars=\\\{\}]
{\color{incolor}In [{\color{incolor}5}]:} \PY{o}{\PYZpc{}\PYZpc{}}\PY{k}{HTML} 
        \PYZlt{}center\PYZgt{}\PYZlt{}video width=\PYZdq{}800\PYZdq{} controls\PYZgt{}\PYZlt{}source src=\PYZdq{}flow\PYZhy{}around\PYZhy{}a\PYZhy{}cylinder.ogg\PYZdq{}\PYZgt{}\PYZlt{}/video\PYZgt{}\PYZlt{}/center\PYZgt{}
\end{Verbatim}

    
    \begin{verbatim}
<IPython.core.display.HTML object>
    \end{verbatim}

    
    \begin{Verbatim}[commandchars=\\\{\}]
{\color{incolor}In [{\color{incolor}6}]:} \PY{k+kn}{from} \PY{n+nn}{IPython}\PY{n+nn}{.}\PY{n+nn}{core}\PY{n+nn}{.}\PY{n+nn}{display} \PY{k}{import} \PY{n}{HTML}
        \PY{k}{def} \PY{n+nf}{css\PYZus{}styling}\PY{p}{(}\PY{p}{)}\PY{p}{:}
            \PY{n}{styles} \PY{o}{=} \PY{n+nb}{open}\PY{p}{(}\PY{l+s+s2}{\PYZdq{}}\PY{l+s+s2}{../styles/custom.css}\PY{l+s+s2}{\PYZdq{}}\PY{p}{,} \PY{l+s+s2}{\PYZdq{}}\PY{l+s+s2}{r}\PY{l+s+s2}{\PYZdq{}}\PY{p}{)}\PY{o}{.}\PY{n}{read}\PY{p}{(}\PY{p}{)}
            \PY{k}{return} \PY{n}{HTML}\PY{p}{(}\PY{n}{styles}\PY{p}{)}
        
        \PY{c+c1}{\PYZsh{} Comment out next line and execute this cell to restore the default notebook style }
        \PY{n}{css\PYZus{}styling}\PY{p}{(}\PY{p}{)}
\end{Verbatim}

            \begin{Verbatim}[commandchars=\\\{\}]
{\color{outcolor}Out[{\color{outcolor}6}]:} <IPython.core.display.HTML object>
\end{Verbatim}
        

    % Add a bibliography block to the postdoc
    
    
    
    \end{document}
