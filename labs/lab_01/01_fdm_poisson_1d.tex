
% Default to the notebook output style

    


% Inherit from the specified cell style.




    
\documentclass[11pt]{article}

    
    
    \usepackage[T1]{fontenc}
    % Nicer default font than Computer Modern for most use cases
    \usepackage{palatino}

    % Basic figure setup, for now with no caption control since it's done
    % automatically by Pandoc (which extracts ![](path) syntax from Markdown).
    \usepackage{graphicx}
    % We will generate all images so they have a width \maxwidth. This means
    % that they will get their normal width if they fit onto the page, but
    % are scaled down if they would overflow the margins.
    \makeatletter
    \def\maxwidth{\ifdim\Gin@nat@width>\linewidth\linewidth
    \else\Gin@nat@width\fi}
    \makeatother
    \let\Oldincludegraphics\includegraphics
    % Set max figure width to be 80% of text width, for now hardcoded.
    \renewcommand{\includegraphics}[1]{\Oldincludegraphics[width=.8\maxwidth]{#1}}
    % Ensure that by default, figures have no caption (until we provide a
    % proper Figure object with a Caption API and a way to capture that
    % in the conversion process - todo).
    \usepackage{caption}
    \DeclareCaptionLabelFormat{nolabel}{}
    \captionsetup{labelformat=nolabel}

    \usepackage{adjustbox} % Used to constrain images to a maximum size 
    \usepackage{xcolor} % Allow colors to be defined
    \usepackage{enumerate} % Needed for markdown enumerations to work
    \usepackage{geometry} % Used to adjust the document margins
    \usepackage{amsmath} % Equations
    \usepackage{amssymb} % Equations
    \usepackage{textcomp} % defines textquotesingle
    % Hack from http://tex.stackexchange.com/a/47451/13684:
    \AtBeginDocument{%
        \def\PYZsq{\textquotesingle}% Upright quotes in Pygmentized code
    }
    \usepackage{upquote} % Upright quotes for verbatim code
    \usepackage{eurosym} % defines \euro
    \usepackage[mathletters]{ucs} % Extended unicode (utf-8) support
    \usepackage[utf8x]{inputenc} % Allow utf-8 characters in the tex document
    \usepackage{fancyvrb} % verbatim replacement that allows latex
    \usepackage{grffile} % extends the file name processing of package graphics 
                         % to support a larger range 
    % The hyperref package gives us a pdf with properly built
    % internal navigation ('pdf bookmarks' for the table of contents,
    % internal cross-reference links, web links for URLs, etc.)
    \usepackage{hyperref}
    \usepackage{longtable} % longtable support required by pandoc >1.10
    \usepackage{booktabs}  % table support for pandoc > 1.12.2
    \usepackage[normalem]{ulem} % ulem is needed to support strikethroughs (\sout)
                                % normalem makes italics be italics, not underlines
    

    
    
    % Colors for the hyperref package
    \definecolor{urlcolor}{rgb}{0,.145,.698}
    \definecolor{linkcolor}{rgb}{.71,0.21,0.01}
    \definecolor{citecolor}{rgb}{.12,.54,.11}

    % ANSI colors
    \definecolor{ansi-black}{HTML}{3E424D}
    \definecolor{ansi-black-intense}{HTML}{282C36}
    \definecolor{ansi-red}{HTML}{E75C58}
    \definecolor{ansi-red-intense}{HTML}{B22B31}
    \definecolor{ansi-green}{HTML}{00A250}
    \definecolor{ansi-green-intense}{HTML}{007427}
    \definecolor{ansi-yellow}{HTML}{DDB62B}
    \definecolor{ansi-yellow-intense}{HTML}{B27D12}
    \definecolor{ansi-blue}{HTML}{208FFB}
    \definecolor{ansi-blue-intense}{HTML}{0065CA}
    \definecolor{ansi-magenta}{HTML}{D160C4}
    \definecolor{ansi-magenta-intense}{HTML}{A03196}
    \definecolor{ansi-cyan}{HTML}{60C6C8}
    \definecolor{ansi-cyan-intense}{HTML}{258F8F}
    \definecolor{ansi-white}{HTML}{C5C1B4}
    \definecolor{ansi-white-intense}{HTML}{A1A6B2}

    % commands and environments needed by pandoc snippets
    % extracted from the output of `pandoc -s`
    \providecommand{\tightlist}{%
      \setlength{\itemsep}{0pt}\setlength{\parskip}{0pt}}
    \DefineVerbatimEnvironment{Highlighting}{Verbatim}{commandchars=\\\{\}}
    % Add ',fontsize=\small' for more characters per line
    \newenvironment{Shaded}{}{}
    \newcommand{\KeywordTok}[1]{\textcolor[rgb]{0.00,0.44,0.13}{\textbf{{#1}}}}
    \newcommand{\DataTypeTok}[1]{\textcolor[rgb]{0.56,0.13,0.00}{{#1}}}
    \newcommand{\DecValTok}[1]{\textcolor[rgb]{0.25,0.63,0.44}{{#1}}}
    \newcommand{\BaseNTok}[1]{\textcolor[rgb]{0.25,0.63,0.44}{{#1}}}
    \newcommand{\FloatTok}[1]{\textcolor[rgb]{0.25,0.63,0.44}{{#1}}}
    \newcommand{\CharTok}[1]{\textcolor[rgb]{0.25,0.44,0.63}{{#1}}}
    \newcommand{\StringTok}[1]{\textcolor[rgb]{0.25,0.44,0.63}{{#1}}}
    \newcommand{\CommentTok}[1]{\textcolor[rgb]{0.38,0.63,0.69}{\textit{{#1}}}}
    \newcommand{\OtherTok}[1]{\textcolor[rgb]{0.00,0.44,0.13}{{#1}}}
    \newcommand{\AlertTok}[1]{\textcolor[rgb]{1.00,0.00,0.00}{\textbf{{#1}}}}
    \newcommand{\FunctionTok}[1]{\textcolor[rgb]{0.02,0.16,0.49}{{#1}}}
    \newcommand{\RegionMarkerTok}[1]{{#1}}
    \newcommand{\ErrorTok}[1]{\textcolor[rgb]{1.00,0.00,0.00}{\textbf{{#1}}}}
    \newcommand{\NormalTok}[1]{{#1}}
    
    % Additional commands for more recent versions of Pandoc
    \newcommand{\ConstantTok}[1]{\textcolor[rgb]{0.53,0.00,0.00}{{#1}}}
    \newcommand{\SpecialCharTok}[1]{\textcolor[rgb]{0.25,0.44,0.63}{{#1}}}
    \newcommand{\VerbatimStringTok}[1]{\textcolor[rgb]{0.25,0.44,0.63}{{#1}}}
    \newcommand{\SpecialStringTok}[1]{\textcolor[rgb]{0.73,0.40,0.53}{{#1}}}
    \newcommand{\ImportTok}[1]{{#1}}
    \newcommand{\DocumentationTok}[1]{\textcolor[rgb]{0.73,0.13,0.13}{\textit{{#1}}}}
    \newcommand{\AnnotationTok}[1]{\textcolor[rgb]{0.38,0.63,0.69}{\textbf{\textit{{#1}}}}}
    \newcommand{\CommentVarTok}[1]{\textcolor[rgb]{0.38,0.63,0.69}{\textbf{\textit{{#1}}}}}
    \newcommand{\VariableTok}[1]{\textcolor[rgb]{0.10,0.09,0.49}{{#1}}}
    \newcommand{\ControlFlowTok}[1]{\textcolor[rgb]{0.00,0.44,0.13}{\textbf{{#1}}}}
    \newcommand{\OperatorTok}[1]{\textcolor[rgb]{0.40,0.40,0.40}{{#1}}}
    \newcommand{\BuiltInTok}[1]{{#1}}
    \newcommand{\ExtensionTok}[1]{{#1}}
    \newcommand{\PreprocessorTok}[1]{\textcolor[rgb]{0.74,0.48,0.00}{{#1}}}
    \newcommand{\AttributeTok}[1]{\textcolor[rgb]{0.49,0.56,0.16}{{#1}}}
    \newcommand{\InformationTok}[1]{\textcolor[rgb]{0.38,0.63,0.69}{\textbf{\textit{{#1}}}}}
    \newcommand{\WarningTok}[1]{\textcolor[rgb]{0.38,0.63,0.69}{\textbf{\textit{{#1}}}}}
    
    
    % Define a nice break command that doesn't care if a line doesn't already
    % exist.
    \def\br{\hspace*{\fill} \\* }
    % Math Jax compatability definitions
    \def\gt{>}
    \def\lt{<}
    % Document parameters
    \title{01\_fdm\_poisson\_1d}
    
    
    

    % Pygments definitions
    
\makeatletter
\def\PY@reset{\let\PY@it=\relax \let\PY@bf=\relax%
    \let\PY@ul=\relax \let\PY@tc=\relax%
    \let\PY@bc=\relax \let\PY@ff=\relax}
\def\PY@tok#1{\csname PY@tok@#1\endcsname}
\def\PY@toks#1+{\ifx\relax#1\empty\else%
    \PY@tok{#1}\expandafter\PY@toks\fi}
\def\PY@do#1{\PY@bc{\PY@tc{\PY@ul{%
    \PY@it{\PY@bf{\PY@ff{#1}}}}}}}
\def\PY#1#2{\PY@reset\PY@toks#1+\relax+\PY@do{#2}}

\expandafter\def\csname PY@tok@nn\endcsname{\let\PY@bf=\textbf\def\PY@tc##1{\textcolor[rgb]{0.00,0.00,1.00}{##1}}}
\expandafter\def\csname PY@tok@sh\endcsname{\def\PY@tc##1{\textcolor[rgb]{0.73,0.13,0.13}{##1}}}
\expandafter\def\csname PY@tok@s\endcsname{\def\PY@tc##1{\textcolor[rgb]{0.73,0.13,0.13}{##1}}}
\expandafter\def\csname PY@tok@kp\endcsname{\def\PY@tc##1{\textcolor[rgb]{0.00,0.50,0.00}{##1}}}
\expandafter\def\csname PY@tok@mo\endcsname{\def\PY@tc##1{\textcolor[rgb]{0.40,0.40,0.40}{##1}}}
\expandafter\def\csname PY@tok@vi\endcsname{\def\PY@tc##1{\textcolor[rgb]{0.10,0.09,0.49}{##1}}}
\expandafter\def\csname PY@tok@go\endcsname{\def\PY@tc##1{\textcolor[rgb]{0.53,0.53,0.53}{##1}}}
\expandafter\def\csname PY@tok@mb\endcsname{\def\PY@tc##1{\textcolor[rgb]{0.40,0.40,0.40}{##1}}}
\expandafter\def\csname PY@tok@gr\endcsname{\def\PY@tc##1{\textcolor[rgb]{1.00,0.00,0.00}{##1}}}
\expandafter\def\csname PY@tok@kn\endcsname{\let\PY@bf=\textbf\def\PY@tc##1{\textcolor[rgb]{0.00,0.50,0.00}{##1}}}
\expandafter\def\csname PY@tok@ne\endcsname{\let\PY@bf=\textbf\def\PY@tc##1{\textcolor[rgb]{0.82,0.25,0.23}{##1}}}
\expandafter\def\csname PY@tok@gs\endcsname{\let\PY@bf=\textbf}
\expandafter\def\csname PY@tok@nv\endcsname{\def\PY@tc##1{\textcolor[rgb]{0.10,0.09,0.49}{##1}}}
\expandafter\def\csname PY@tok@cm\endcsname{\let\PY@it=\textit\def\PY@tc##1{\textcolor[rgb]{0.25,0.50,0.50}{##1}}}
\expandafter\def\csname PY@tok@s1\endcsname{\def\PY@tc##1{\textcolor[rgb]{0.73,0.13,0.13}{##1}}}
\expandafter\def\csname PY@tok@gi\endcsname{\def\PY@tc##1{\textcolor[rgb]{0.00,0.63,0.00}{##1}}}
\expandafter\def\csname PY@tok@sx\endcsname{\def\PY@tc##1{\textcolor[rgb]{0.00,0.50,0.00}{##1}}}
\expandafter\def\csname PY@tok@nt\endcsname{\let\PY@bf=\textbf\def\PY@tc##1{\textcolor[rgb]{0.00,0.50,0.00}{##1}}}
\expandafter\def\csname PY@tok@o\endcsname{\def\PY@tc##1{\textcolor[rgb]{0.40,0.40,0.40}{##1}}}
\expandafter\def\csname PY@tok@sb\endcsname{\def\PY@tc##1{\textcolor[rgb]{0.73,0.13,0.13}{##1}}}
\expandafter\def\csname PY@tok@ch\endcsname{\let\PY@it=\textit\def\PY@tc##1{\textcolor[rgb]{0.25,0.50,0.50}{##1}}}
\expandafter\def\csname PY@tok@gh\endcsname{\let\PY@bf=\textbf\def\PY@tc##1{\textcolor[rgb]{0.00,0.00,0.50}{##1}}}
\expandafter\def\csname PY@tok@se\endcsname{\let\PY@bf=\textbf\def\PY@tc##1{\textcolor[rgb]{0.73,0.40,0.13}{##1}}}
\expandafter\def\csname PY@tok@nb\endcsname{\def\PY@tc##1{\textcolor[rgb]{0.00,0.50,0.00}{##1}}}
\expandafter\def\csname PY@tok@il\endcsname{\def\PY@tc##1{\textcolor[rgb]{0.40,0.40,0.40}{##1}}}
\expandafter\def\csname PY@tok@s2\endcsname{\def\PY@tc##1{\textcolor[rgb]{0.73,0.13,0.13}{##1}}}
\expandafter\def\csname PY@tok@kd\endcsname{\let\PY@bf=\textbf\def\PY@tc##1{\textcolor[rgb]{0.00,0.50,0.00}{##1}}}
\expandafter\def\csname PY@tok@err\endcsname{\def\PY@bc##1{\setlength{\fboxsep}{0pt}\fcolorbox[rgb]{1.00,0.00,0.00}{1,1,1}{\strut ##1}}}
\expandafter\def\csname PY@tok@cs\endcsname{\let\PY@it=\textit\def\PY@tc##1{\textcolor[rgb]{0.25,0.50,0.50}{##1}}}
\expandafter\def\csname PY@tok@nd\endcsname{\def\PY@tc##1{\textcolor[rgb]{0.67,0.13,1.00}{##1}}}
\expandafter\def\csname PY@tok@nc\endcsname{\let\PY@bf=\textbf\def\PY@tc##1{\textcolor[rgb]{0.00,0.00,1.00}{##1}}}
\expandafter\def\csname PY@tok@ow\endcsname{\let\PY@bf=\textbf\def\PY@tc##1{\textcolor[rgb]{0.67,0.13,1.00}{##1}}}
\expandafter\def\csname PY@tok@mi\endcsname{\def\PY@tc##1{\textcolor[rgb]{0.40,0.40,0.40}{##1}}}
\expandafter\def\csname PY@tok@k\endcsname{\let\PY@bf=\textbf\def\PY@tc##1{\textcolor[rgb]{0.00,0.50,0.00}{##1}}}
\expandafter\def\csname PY@tok@c\endcsname{\let\PY@it=\textit\def\PY@tc##1{\textcolor[rgb]{0.25,0.50,0.50}{##1}}}
\expandafter\def\csname PY@tok@kt\endcsname{\def\PY@tc##1{\textcolor[rgb]{0.69,0.00,0.25}{##1}}}
\expandafter\def\csname PY@tok@no\endcsname{\def\PY@tc##1{\textcolor[rgb]{0.53,0.00,0.00}{##1}}}
\expandafter\def\csname PY@tok@kr\endcsname{\let\PY@bf=\textbf\def\PY@tc##1{\textcolor[rgb]{0.00,0.50,0.00}{##1}}}
\expandafter\def\csname PY@tok@gd\endcsname{\def\PY@tc##1{\textcolor[rgb]{0.63,0.00,0.00}{##1}}}
\expandafter\def\csname PY@tok@vg\endcsname{\def\PY@tc##1{\textcolor[rgb]{0.10,0.09,0.49}{##1}}}
\expandafter\def\csname PY@tok@nl\endcsname{\def\PY@tc##1{\textcolor[rgb]{0.63,0.63,0.00}{##1}}}
\expandafter\def\csname PY@tok@bp\endcsname{\def\PY@tc##1{\textcolor[rgb]{0.00,0.50,0.00}{##1}}}
\expandafter\def\csname PY@tok@nf\endcsname{\def\PY@tc##1{\textcolor[rgb]{0.00,0.00,1.00}{##1}}}
\expandafter\def\csname PY@tok@si\endcsname{\let\PY@bf=\textbf\def\PY@tc##1{\textcolor[rgb]{0.73,0.40,0.53}{##1}}}
\expandafter\def\csname PY@tok@kc\endcsname{\let\PY@bf=\textbf\def\PY@tc##1{\textcolor[rgb]{0.00,0.50,0.00}{##1}}}
\expandafter\def\csname PY@tok@sd\endcsname{\let\PY@it=\textit\def\PY@tc##1{\textcolor[rgb]{0.73,0.13,0.13}{##1}}}
\expandafter\def\csname PY@tok@m\endcsname{\def\PY@tc##1{\textcolor[rgb]{0.40,0.40,0.40}{##1}}}
\expandafter\def\csname PY@tok@ge\endcsname{\let\PY@it=\textit}
\expandafter\def\csname PY@tok@mf\endcsname{\def\PY@tc##1{\textcolor[rgb]{0.40,0.40,0.40}{##1}}}
\expandafter\def\csname PY@tok@vc\endcsname{\def\PY@tc##1{\textcolor[rgb]{0.10,0.09,0.49}{##1}}}
\expandafter\def\csname PY@tok@c1\endcsname{\let\PY@it=\textit\def\PY@tc##1{\textcolor[rgb]{0.25,0.50,0.50}{##1}}}
\expandafter\def\csname PY@tok@sr\endcsname{\def\PY@tc##1{\textcolor[rgb]{0.73,0.40,0.53}{##1}}}
\expandafter\def\csname PY@tok@na\endcsname{\def\PY@tc##1{\textcolor[rgb]{0.49,0.56,0.16}{##1}}}
\expandafter\def\csname PY@tok@sc\endcsname{\def\PY@tc##1{\textcolor[rgb]{0.73,0.13,0.13}{##1}}}
\expandafter\def\csname PY@tok@gt\endcsname{\def\PY@tc##1{\textcolor[rgb]{0.00,0.27,0.87}{##1}}}
\expandafter\def\csname PY@tok@mh\endcsname{\def\PY@tc##1{\textcolor[rgb]{0.40,0.40,0.40}{##1}}}
\expandafter\def\csname PY@tok@gu\endcsname{\let\PY@bf=\textbf\def\PY@tc##1{\textcolor[rgb]{0.50,0.00,0.50}{##1}}}
\expandafter\def\csname PY@tok@cp\endcsname{\def\PY@tc##1{\textcolor[rgb]{0.74,0.48,0.00}{##1}}}
\expandafter\def\csname PY@tok@ni\endcsname{\let\PY@bf=\textbf\def\PY@tc##1{\textcolor[rgb]{0.60,0.60,0.60}{##1}}}
\expandafter\def\csname PY@tok@ss\endcsname{\def\PY@tc##1{\textcolor[rgb]{0.10,0.09,0.49}{##1}}}
\expandafter\def\csname PY@tok@gp\endcsname{\let\PY@bf=\textbf\def\PY@tc##1{\textcolor[rgb]{0.00,0.00,0.50}{##1}}}
\expandafter\def\csname PY@tok@w\endcsname{\def\PY@tc##1{\textcolor[rgb]{0.73,0.73,0.73}{##1}}}
\expandafter\def\csname PY@tok@cpf\endcsname{\let\PY@it=\textit\def\PY@tc##1{\textcolor[rgb]{0.25,0.50,0.50}{##1}}}

\def\PYZbs{\char`\\}
\def\PYZus{\char`\_}
\def\PYZob{\char`\{}
\def\PYZcb{\char`\}}
\def\PYZca{\char`\^}
\def\PYZam{\char`\&}
\def\PYZlt{\char`\<}
\def\PYZgt{\char`\>}
\def\PYZsh{\char`\#}
\def\PYZpc{\char`\%}
\def\PYZdl{\char`\$}
\def\PYZhy{\char`\-}
\def\PYZsq{\char`\'}
\def\PYZdq{\char`\"}
\def\PYZti{\char`\~}
% for compatibility with earlier versions
\def\PYZat{@}
\def\PYZlb{[}
\def\PYZrb{]}
\makeatother


    % Exact colors from NB
    \definecolor{incolor}{rgb}{0.0, 0.0, 0.5}
    \definecolor{outcolor}{rgb}{0.545, 0.0, 0.0}



    
    % Prevent overflowing lines due to hard-to-break entities
    \sloppy 
    % Setup hyperref package
    \hypersetup{
      breaklinks=true,  % so long urls are correctly broken across lines
      colorlinks=true,
      urlcolor=urlcolor,
      linkcolor=linkcolor,
      citecolor=citecolor,
      }
    % Slightly bigger margins than the latex defaults
    
    \geometry{verbose,tmargin=1in,bmargin=1in,lmargin=1in,rmargin=1in}
    
    

    \begin{document}
    
    
    \maketitle
    
    

    
    \section{Homework 1}\label{homework-1}

    \subsection{General Instructions}\label{general-instructions}

\begin{itemize}
\item
  To pass this assignment you are asked to submit both the program
  sources and a short report including results, discussion etc.
\item
  For the theoretical exercises, please include intermediate steps to
  explain how you arrive at your solution.
\item
  For the computational exercises, you can use either Matlab, Octave (an
  open source variant of Matlab) or Python to implement your computer
  programs.
\item
  Please submit the \textbf{complete} computer program or script you
  used to solve the computational problems. Don't overengineer your
  code, keep it as simple and readable as possible and provide short
  code comments to help other people understanding your code.
\item
  Please provide also a short summary and discussion of your results
  including the requested output (e.g.~tables, graphs etc).
\item
  You have two options to submit you solution. Either you submit the
  program source and the report in a separate pdf file or you can
  include everything in a single jupyter notebook, preferable based on
  the original homework notebook.
\item
  Up to 3 students can jointly submit the solutions. \textbf{Only 1
  student from each group} is supposed to submit them. Please indicate
  the other members of your group in the comment field appearing when
  you submit your files in Cambro.
\item
  Deadline for submission of your solutions is \textbf{19th of
  November}.
\end{itemize}

\subparagraph{Happy coding!}\label{happy-coding}

    \subsection{Exercise 1}\label{exercise-1}

Find at least 4 more ``famous'' partial differential equations (PDE)
with one in each category ``Linear PDE, Non-linear PDE, Linear System,
Non-linear system''. Give a brief description of the underlying
phenomena modeled by the PDE.

    \subsection{Problem 2}\label{problem-2}

In Lecture 2, we introduced the \textbf{central finite difference
operators} \[
\partial^0 u(x)
=  \dfrac{u(x+h) - u(x-h)}
{2h}
\approx u'(x)
\] and \[
\partial^+ \partial^- u(x)
=  \dfrac{u(x+h) - 2 u(x) + u(x-h)}
{h^2}
\approx u''(x)
\] as an approximation of the first and the second order derivative
\(u'(x)\) and \(u''(x)\), respectively.

Recall that for \(u \in C^k([0,1])\), the Taylor expansion of \(u\)
around \(x\) is given by \[
u(x+h) = u(x) + h u'(x) + \dfrac{h^2}{2!} u''(x) 
%+\dfrac{h^3}{3!} u^{(3)}(x)
+ \ldots 
+ \dfrac{h^{k-1}}{(k-1)!} u^{(k-1)}(x)
+ \dfrac{h^{k}}{k!} u^{(k)}(\xi)
\] for some \(\xi \in (x,x+h)\). Since \(u \in C^k([0,1])\), the
remainder term \(\dfrac{h^k}{k!} u^{(k)}(\xi)\) is uniformly bounded
with respect to \(\xi\) and thus we can simply write \[
u(x+h) = u(x) + h u'(x) + \dfrac{h^2}{2!} u''(x) 
%+\dfrac{h^3}{3!} u^{(3)}(x)
+ \ldots 
+ \dfrac{h^{k-1}}{(k-1)!} u^{(k-1)}(x)
+ \mathcal{O}(h^k)
\]

\textbf{a)} Use Taylor expansion to show that for \(u \in C^3([0,1])\)
\[
\| \partial^0u(x) - u'(x) \|_{C([0,1])}
\leqslant 
C h^2
 \| u^{(3)} \|_{C([0,1])}
\]

\textbf{b)} Similarly, demonstrate that \[
\| \partial^+ \partial^-u(x) - u''(x) \|_{C([0,1])}
\leqslant 
C h^2
 \| u^{(4)} \|_{C([0,1])}
\] assuming that \(u \in C^4([0,1])\).

    \subsection{Computational Problem 1}\label{computational-problem-1}

In this problem set you are asked to solve the Poisson problem \[
- u''  = f \quad \text{in } (0,1)
\] numerically for various types of boundary conditions.

\textbf{a)} Start with implementing the finite difference method (FDM)
from Lecture 2 using the right-hand side \[
f = \sin(2 \pi x)
\] and boundary conditions \[u(0) = u(1) = 0.\] Plot the solutions for
different mesh sizes \(h = 1/N\) with \(N = 4, 8, 16, 32, 64\) in the
\emph{same} plot. Find the \emph{exact} analytical solution \(u\) to the
given Poisson problem (Hint: it should be very similar to \(f\)) and
plot it for \(N = 64\) into the same figure. Does your computed discrete
solution \(U\) converge to \(u\)?

\textbf{b)} Next, we switch to a Neumann boundary condition on the left
endpoint; that is \[ -u'(0) = \sigma_0, \quad u(1) = 0.\] Modify your
FDM solver to incoporate the Neumann condition based on the one-side
apprioximation \[ -u'(0)\approx \frac{u(0) - u(h)}{h} = \sigma.\] Take
the exact solution \(u\) from a) and calculate its proper value for
\(\sigma\). Conduct a similar numerical study as in part a). What do you
observe regarding the accuracy of the method?

\textbf{c)} Now try to solve the same Poisson problem, now with boundary
conditions \[ -u'(0) = \sigma_0,\quad u'(1) = \sigma_1\] using values
\(\sigma_0\) and \(\sigma_1\) corresponding to the exact solution \(u\)
in part a). What happens when you try to solve the linear algebra
system? Why?

    \subsubsection{Useful code snippets}\label{useful-code-snippets}

As most of you are familiar with MATLAB but not so familar with Python,
we provide a number of code snippets to get you started in Python. Three
dots \(\ldots\) indicate places where you have to fill in code. Note
that this outline provides only a very rudimentary inefficient
implementation to begin with and we will refine our methods while
progressing towards more advanced and larger problems.

We start with importing the necessary scientific libraries and define a
name alias for them.

    \begin{Verbatim}[commandchars=\\\{\}]
{\color{incolor}In [{\color{incolor}3}]:} \PY{c+c1}{\PYZsh{} Arrary and stuff }
        \PY{k+kn}{import} \PY{n+nn}{numpy} \PY{k}{as} \PY{n+nn}{np}
        \PY{c+c1}{\PYZsh{} Linear algebra solvers from scipy}
        \PY{k+kn}{import} \PY{n+nn}{scipy}\PY{n+nn}{.}\PY{n+nn}{linalg} \PY{k}{as} \PY{n+nn}{la}
        \PY{c+c1}{\PYZsh{} Basic plotting routines from the matplotlib library }
        \PY{k+kn}{import} \PY{n+nn}{matplotlib}\PY{n+nn}{.}\PY{n+nn}{pyplot} \PY{k}{as} \PY{n+nn}{plt}
\end{Verbatim}

    Next we define the grid points.

    \begin{Verbatim}[commandchars=\\\{\}]
{\color{incolor}In [{\color{incolor}4}]:} \PY{c+c1}{\PYZsh{} Number of equally spaced subintervals}
        \PY{n}{N} \PY{o}{=} \PY{l+m+mi}{4}
        \PY{c+c1}{\PYZsh{} Mesh size}
        \PY{n}{h} \PY{o}{=} \PY{l+m+mi}{1}\PY{o}{/}\PY{n}{N} \PY{c+c1}{\PYZsh{}Important! In Python 2 you needed to write 1.0 to prevent integer divsion}
        \PY{c+c1}{\PYZsh{} Define N+1 grid points via linspace which is part of numpy now aliased as np }
        \PY{n}{x} \PY{o}{=} \PY{n}{np}\PY{o}{.}\PY{n}{linspace}\PY{p}{(}\PY{l+m+mi}{0}\PY{p}{,}\PY{l+m+mi}{1}\PY{p}{,}\PY{n}{N}\PY{o}{+}\PY{l+m+mi}{1}\PY{p}{)}
        \PY{n+nb}{print}\PY{p}{(}\PY{n}{x}\PY{p}{)}
\end{Verbatim}

    \begin{Verbatim}[commandchars=\\\{\}]
[ 0.    0.25  0.5   0.75  1.  ]

    \end{Verbatim}

    Now define matrix \(A\) and right-hand side vector \(F\). We will first
fill in the values that will be unchanged for different boundary
conditions.

    \begin{Verbatim}[commandchars=\\\{\}]
{\color{incolor}In [{\color{incolor}93}]:} \PY{c+c1}{\PYZsh{} Define a (full) matrix filled with 0s.}
         \PY{n}{A} \PY{o}{=} \PY{n}{np}\PY{o}{.}\PY{n}{zeros}\PY{p}{(}\PY{p}{(}\PY{n}{N}\PY{o}{+}\PY{l+m+mi}{1}\PY{p}{,} \PY{n}{N}\PY{o}{+}\PY{l+m+mi}{1}\PY{p}{)}\PY{p}{)}
         
         \PY{c+c1}{\PYZsh{} Define tridiagonal part of A by for rows 1 to N\PYZhy{}1}
         \PY{k}{for} \PY{n}{i} \PY{o+ow}{in} \PY{n+nb}{range}\PY{p}{(}\PY{l+m+mi}{1}\PY{p}{,} \PY{n}{N}\PY{p}{)}\PY{p}{:}
             \PY{n}{A}\PY{p}{[}\PY{n}{i}\PY{p}{,} \PY{n}{i}\PY{o}{\PYZhy{}}\PY{l+m+mi}{1}\PY{p}{]} \PY{o}{=} \PY{o}{.}\PY{o}{.}\PY{o}{.}
             \PY{n}{A}\PY{p}{[}\PY{n}{i}\PY{p}{,} \PY{n}{i}\PY{o}{+}\PY{l+m+mi}{1}\PY{p}{]} \PY{o}{=} \PY{o}{.}\PY{o}{.}\PY{o}{.}
             \PY{n}{A}\PY{p}{[}\PY{n}{i}\PY{p}{,} \PY{n}{i}\PY{p}{]} \PY{o}{=} \PY{o}{.}\PY{o}{.}\PY{o}{.}
             
         \PY{c+c1}{\PYZsh{} Define right hand side. Instead of iterating we}
         \PY{c+c1}{\PYZsh{} use a vectorized variant to evaluate f on all grid points}
         \PY{c+c1}{\PYZsh{} Look out for the right h factors! }
         \PY{n}{F} \PY{o}{=} \PY{o}{.}\PY{o}{.}\PY{o}{.}\PY{o}{*}\PY{n}{np}\PY{o}{.}\PY{n}{sin}\PY{p}{(}\PY{l+m+mi}{2}\PY{o}{*}\PY{n}{np}\PY{o}{.}\PY{n}{pi}\PY{o}{*}\PY{n}{x}\PY{p}{)}
         
         \PY{c+c1}{\PYZsh{} Note that F[0] and F[N] are also filled!}
\end{Verbatim}

    Last step to set up the system is to take the boundary conditions into
account by modifying \(A\) and \(F\) properly.

    \begin{Verbatim}[commandchars=\\\{\}]
{\color{incolor}In [{\color{incolor} }]:} \PY{c+c1}{\PYZsh{} Left boundary}
        \PY{n}{A}\PY{p}{[}\PY{l+m+mi}{0}\PY{p}{,}\PY{l+m+mi}{0}\PY{p}{]} \PY{o}{=} \PY{o}{.}\PY{o}{.}\PY{o}{.}
        \PY{n}{F}\PY{p}{[}\PY{l+m+mi}{0}\PY{p}{]} \PY{o}{=} \PY{o}{.}\PY{o}{.}\PY{o}{.}
        
        \PY{c+c1}{\PYZsh{} Right boundary}
        \PY{n}{A}\PY{p}{[}\PY{n}{N}\PY{p}{,}\PY{n}{N}\PY{p}{]} \PY{o}{=} \PY{o}{.}\PY{o}{.}\PY{o}{.}
        \PY{n}{F}\PY{p}{[}\PY{n}{N}\PY{p}{]} \PY{o}{=} \PY{o}{.}\PY{o}{.}\PY{o}{.}
\end{Verbatim}

    Now we solve the linear algebra system \(AU = F\) and plot the results.

    \begin{Verbatim}[commandchars=\\\{\}]
{\color{incolor}In [{\color{incolor} }]:} \PY{n}{U} \PY{o}{=} \PY{n}{la}\PY{o}{.}\PY{n}{solve}\PY{p}{(}\PY{n}{A}\PY{p}{,} \PY{n}{F}\PY{p}{)}
        \PY{c+c1}{\PYZsh{}  \PYZdq{}x\PYZhy{}r\PYZdq{} means mark data points as \PYZdq{}x\PYZdq{}, connect them by a line and use red color}
        \PY{n}{plt}\PY{o}{.}\PY{n}{plot}\PY{p}{(}\PY{n}{x}\PY{p}{,} \PY{n}{U}\PY{p}{,} \PY{l+s+s2}{\PYZdq{}}\PY{l+s+s2}{x\PYZhy{}r}\PY{l+s+s2}{\PYZdq{}}\PY{p}{)}
\end{Verbatim}

    With these snippets in place you should be able to solve Computer
Problem 1 but don't hesitate to ask if you are wondering about
something!

    \subsection{Computational Problem 2}\label{computational-problem-2}

The goal of this problem is to investigate the numerical error
introduced by the FDM more quantitatively and to familiarize us with the
\textbf{method of manufactured solution}.

The idea is to assess the accurracy and correctness of a PDE solver
implementation by constructing a know reference solution which solves
the PDE problem at hand. This can be simply done by picking a meaningful
and not to boring analytical solution and explicitly calculate the data
which need to be supplied, e.g., the right-hand side or boundary values
for various boundary condition.

For instance, taking the function \(u(x) = x + \sin(2 \pi x)\), we can
simply calculate that

\begin{align}
 u'(x) &= 1 + 2\pi \cos(2\pi x)
 \\
 u''(x) &= - (2\pi)^2 \sin(2\pi x)
\end{align}

and thus \(u\) satisfies the Poisson problem \[
-u''(x) = (2\pi)^2 \sin(2\pi x)
\] with boundary conditions \[
-u'(0) = -(1+2\pi), \quad u(1) = 1.
\]

With a known reference solution at hand we can compute the error vector
\(u_i - U_i\) at the grid points \(\{x_i\}_{i=0}^N\) for a series of
successively refined grid, e.g.~by taking \(N = 4\cdot2^k\) for
\(k = 0, 1, 2, 3\) etc.

Reducing the mesh size \(h\) by half allows us to easily compute the
\textbf{experimental order of convergence (EOC)}, that is the observed
error reduction in the numerical solution when passing from a coarser
mesh with mesh size \(h\) to to a finer mesh with \(h/2\). The EOC can
then be compare with the theoretically predicted error reduction (if
known).

For instance, if you know that the discretization error \(E(h)\) given
on mesh with mesh size \(h\) and measured in some norm $ \textbar{}
\cdot \textbar{}$ behaves like \(\| E(h) \| \sim h^k\), you can
conclude that\\
\[
\dfrac{\|E(h/2)\|}{\|E(h)\|} \sim \dfrac{(h/2)^k}{h^k} = (1/2)^k
\] when passing from \(h\) to half the mesh size \(h/2\). Taking the
logarithm of the last equation shows \[
k \sim \dfrac{\ln(\|E(h)\|/\|E(h/2)\|}{\ln 2}. 
\] (Verify this!) Thus the EOC is measured by the \[
EOC = \dfrac{\ln(\|E(h)\|/\|E(h/2)\|}{\ln 2}.
\]

Alternatively, you can do a log-log plot of your error as function of
\(h\), that is plot \(\ln (\| E(h)\|)\) againts \(\ln(h)\). Then the
slope of this plot should be \(\sim k\) if we expect the method to be of
convergence order \(k\)

\begin{enumerate}
\def\labelenumi{\alph{enumi})}
\item
  Use this approach to verify your FDM program developed in
  Computational Problem 1 a) by computing the error for
  \(N= 4, 8, 16, 32, 64\) in the maximum norm. Give the corresponding
  log-log plot and report your convergence order. Do you achieve 2nd
  order convergence?
\item
  Next, repeat the same experiment for the Poisson problem 1b) with
  mixed Dirchlet/Neumann boundary conditions. What EOC do you observe?
  Can you explain it?
\end{enumerate}

    \subsection{Computational Problem 3}\label{computational-problem-3}

In the final computer exercise you are asked to extend your FDM solver
in order to compute a solution to the \emph{Convection-Diffusion
problem} \[
- \epsilon u''(x) + b u'(x) = f(x) \quad \text{for } x \in (0,1),
\\
u(0) = u(1) = 0,
\] with \(b = 1\) and various \(\epsilon\) tending \(0\). While for
\(\epsilon > 0\), the problem is clearly a 2nd order problem, its
characteristics change drastically for \(\epsilon \to 0\). Formally, the
limit equation is given by the \textbf{first order} problem \[
b u'(x) = f(x) \quad \text{for } x \in (0,1)
\] and we see immediately that only \emph{one} boundary condition should
be required in the limit case. (Convince yourself by assuming that
\(f = 1\) and trying to compute a solution). It turns out that it is
natural to impose a Dirichlet boundary condition \(u(0) = u_0\) only at
the ``inflow point'' \(x(0)\) and thus the ``outflow point''
\(u(1) = u_1\) becomes ``superfluous'' when \(\epsilon \to 0\). Here, we
will study what happens to our FDM solver when we gradually approache
this limit case.

\textbf{a)} Compute \(f\) such that

\begin{align}
  u(x)
  = 
  x - 
  \dfrac{e^{(x-1)/\epsilon} - e^{-1/\epsilon}}
  {1 - e^{-1/\epsilon}}
\end{align}

is an exact solution for \(b = 1\) and arbitrary \(\epsilon\) (Hint:
\(f\) should not look too complicated\ldots{}).

\textbf{b)} Start with using the symmetric/central difference operator
\[
\partial^0 U_i = \dfrac{U_{i+1} - U_{i-1}}{2h}
\] to discretize the first order derivative \(b u'(x)\). How does the
resulting matrix system look like? Adapt your FDM solver from Problem 1
accordingly and verify your implementation employing the method of
manufactured solution from Problem 2.

\textbf{c)} Now repeat the numerical experiment and compute a numerical
solution \(U_{\epsilon}\) for \(\epsilon = 0.1, 0.01. 0.001\) and at
least \(4\) successively refined grids. For each \(\epsilon\) provide a
plot including the exact solution and the computed approximations. What
do you observe?

Report the EOC, this time in a 2 column table with the first column
reporting the mesh size and the second column reporting the computed
error. Is the error reduced by a factor \(4\) as expected from a 2nd
order convergent method? Can you recover order \(2\) by making the
meshes even finer?

\textbf{d}) Finally, again, conduct the same experiment after replacing
\(\partial^0\) by 1) \(\partial^+\) and 2) \(\partial ^-\). Describe
your observations of the discrete solution behavior. Which variant gives
the most satisfying/robust solution for small \(\epsilon\)?

Finally compute again the EOC. What do you get?

    The following cell loads non-default styles for the notebook

    \begin{Verbatim}[commandchars=\\\{\}]
{\color{incolor}In [{\color{incolor}2}]:} \PY{k+kn}{from} \PY{n+nn}{IPython}\PY{n+nn}{.}\PY{n+nn}{core}\PY{n+nn}{.}\PY{n+nn}{display} \PY{k}{import} \PY{n}{HTML}
        \PY{k}{def} \PY{n+nf}{css\PYZus{}styling}\PY{p}{(}\PY{p}{)}\PY{p}{:}
            \PY{c+c1}{\PYZsh{}styles = open(\PYZdq{}../styles/custom.css\PYZdq{}, \PYZdq{}r\PYZdq{}).read()}
            \PY{n}{styles} \PY{o}{=} \PY{n+nb}{open}\PY{p}{(}\PY{l+s+s2}{\PYZdq{}}\PY{l+s+s2}{../styles/numericalmoocstyle.css}\PY{l+s+s2}{\PYZdq{}}\PY{p}{,} \PY{l+s+s2}{\PYZdq{}}\PY{l+s+s2}{r}\PY{l+s+s2}{\PYZdq{}}\PY{p}{)}\PY{o}{.}\PY{n}{read}\PY{p}{(}\PY{p}{)}
            \PY{k}{return} \PY{n}{HTML}\PY{p}{(}\PY{n}{styles}\PY{p}{)}
        
        \PY{c+c1}{\PYZsh{} Comment out next line and execute this cell to restore the default notebook style }
        \PY{n}{css\PYZus{}styling}\PY{p}{(}\PY{p}{)}
\end{Verbatim}

            \begin{Verbatim}[commandchars=\\\{\}]
{\color{outcolor}Out[{\color{outcolor}2}]:} <IPython.core.display.HTML object>
\end{Verbatim}
        

    % Add a bibliography block to the postdoc
    
    
    
    \end{document}
